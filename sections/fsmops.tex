\section{FSM operations}
\label{sec:fsmops}

We describe here some operations over FSMs. A FSM $\mathcal{M}_i$ is
defined as:
\begin{equation}
    \mathcal{M}_i = (Q_i, \Sigma_i, K_i, L_i, \boldsymbol{\alpha}_i,
        \mathbf{T}_i, \boldsymbol{\omega}_i, \boldsymbol{\lambda}_i).
\end{equation}
The path weight, path label and input sequencs weight function
associated to this FSM are denoted $\mu_i(\cdot)$, $\sigma_i(\cdot)$ and
$\nu_i(\cdot)$ respectively. The set of input sequences accepted by
$\mathcal{M}_i$ is $S_i = \{\mathbf{s}: \nu_i(\mathbf{s}) > 0 \}$. In
the following, for binary operator over 2 FSMs, we assume tacitly that
both FSMs have the same set of symbols, i.e. $\Sigma = \Sigma_1 = \Sigma_2$
and the same weight semiring, i.e. $K = K_1 = K_2$, and the same
label free monoid, i.e. $L = L_1 = L_2$.

\subsection{Renormalization}

We say that a FSM $\mathcal{M}$ is \emph{normalized} if (i) the sum
of its intial weights sum up to 1, i.e. $\sum_{i \in Q} \alpha_i = 1$,
and (ii) the sum of the transition weights from a state $i$ to all
other states and the final weight of the
state $i$ sum up to 1, i.e. $\omega_i + \sum_{j \in Q} T_{ij} = 1$.
A FSM $\mathcal{M}_1$ with a zero-sum free and divisble weight
semiring can be transformed into normalized FSM $\mathcal{M}_2$ via
the renormalization operation: $\mathcal{M}_2 = \text{renorm}(\mathcal{M}_1)$.
The resulting FSM is obtained by the following construction:
\begin{align}
    Q_2 &= Q_1 & \boldsymbol{\alpha}_2 &=
        \Big(\sum_i \alpha_{1i} \Big)^{-1}\boldsymbol{\alpha}_{1}  \\
        \mathbf{T}_2 &= \begin{bmatrix}
            (\omega_1 + \sum_i T_{1,1i})^{-1} \mathbf{T}_{1,1} \\
            \vdots \\
            (\omega_d + \sum_i T_{1,di})^{-1} \mathbf{T}_{1,d} \\
        \end{bmatrix} & \boldsymbol{\omega}_2 &= \begin{bmatrix}
        (\omega_1 + \sum_i T_{1,1i})^{-1} \\
        \vdots \\
        (\omega_d + \sum_i T_{1,di})^{-1} \\
        \end{bmatrix} \odot \boldsymbol{\omega}_1 \\
        \boldsymbol{\lambda}_2 &= \boldsymbol{\lambda}_1 & &
\end{align}


\subsection{Union}

The union of two FSMs $\mathcal{M}_1 \cup \mathcal{M}_2$ gives a FSM
$\mathcal{M}_3$ such that $S_3 = S_1 \cup S_2$ and
$\forall \mathbf{s} \in S_3$, $\nu_3(\mathbf{s}) =
\nu_1(\mathbf{s}) + \nu_2(\mathbf{s})$.
$\mathcal{M}_3$ can be obtained with the following construction:
\begin{align}
    Q_3 &= \{1, \dots, |Q_1| + |Q_2| \} &
    \boldsymbol{\alpha}_3 &= \begin{bmatrix}
        \boldsymbol{\alpha}_1 \\
        \boldsymbol{\alpha}_2
    \end{bmatrix} \\
    \mathbf{T}_3 &= \begin{bmatrix}
        \mathbf{T}_1 & \\
        & \mathbf{T}_2
    \end{bmatrix} &
    \boldsymbol{\omega}_3 &= \begin{bmatrix}
        \boldsymbol{\omega}_1 \\
        \boldsymbol{\omega}_2
    \end{bmatrix} \\
    \boldsymbol{\lambda}_3 &= \begin{bmatrix}
        \boldsymbol{\lambda}_1 \\
        \boldsymbol{\lambda}_2
    \end{bmatrix} & &
\end{align}

\subsection{Concatenation}

The concatenation of two FSMs $\text{concat}({M}_1, \mathcal{M}_2)$
gives a FSM $\mathcal{M}_3$ such that $S_3 = \{ \mathbf{s}_1 \mathbf{s}_2 :
\mathbf{s}_1 \in S_1, \ \mathbf{s}_2 \in S_2 \}$. $\mathcal{M}_3$
can be obtained with the following construction:
\begin{align}
    Q_3 &= \{1, \dots, |Q_1| + |Q_2| \} &
    \boldsymbol{\alpha}_3 &= \begin{bmatrix}
        \boldsymbol{\alpha}_1 \\
        0 \boldsymbol{\alpha}_2
    \end{bmatrix} \\
    \mathbf{T}_3 &= \begin{bmatrix}
        \mathbf{T}_1 & \boldsymbol{\omega}_1 \boldsymbol{\alpha}_2^\top \\
        & \mathbf{T}_2
    \end{bmatrix} &
    \boldsymbol{\omega}_3 &= \begin{bmatrix}
        0 \boldsymbol{\omega}_1 \\
        \boldsymbol{\omega}_2
    \end{bmatrix} \\
    \boldsymbol{\lambda}_3 &= \begin{bmatrix}
        \boldsymbol{\lambda}_1 \\
        \boldsymbol{\lambda}_2
    \end{bmatrix} & &
\end{align}

\subsection{Reversal}

The reversal (denoted $^\top$) of a FSM $\mathcal{M}_1$ yields a FSM
$\mathcal{M}_2 = \mathcal{M}_1^\top$ such that $S_2 = \{ \overleftarrow{\mathbf{s}} :
\mathbf{s} \in S_1 \}$ where $\overleftarrow{\mathbf{s}}$ is the sequence
$\mathbf{s}$ in reversed order. $\mathcal{M}_2$ is obtained by the
following construction:
\begin{align}
    Q_2 &= Q_1 &
    \boldsymbol{\alpha}_2 &= \boldsymbol{\omega}_1 \\
    \mathbf{T}_2 &= \mathbf{T}_1^\top &
    \boldsymbol{\omega}_2 &= \boldsymbol{\alpha}_1 \\
        \boldsymbol{\lambda}_2 &= \boldsymbol{\lambda}_1 & &
\end{align}

\subsection{Composition}

Let's consider a FSM $\mathcal{M}_1$ with $d$ states. We would like to
compose $\mathcal{M}_1$ with a sequence of $d$ FSMs
$\mathcal{M}^{1:d} = (\mathcal{M}^1, \dots, \mathcal{M}^d)$
That is to say that the $i$th state of $\mathcal{M}_1$ is replaced
with the FSM $\mathcal{M}^i$. The composition of $\mathcal{M}_1$ and
$\mathcal{M}^{1:d}$, denoted $\mathcal{M}_1 \circ \mathcal{M}^{1:d}$,
yields a FSM $\mathcal{M}_2$ which can be obtained with the following
construction:
\begin{align}
    Q_2 &= \{1, \dots, \sum_{i=1}^d |Q^i| \} \\
    \boldsymbol{\alpha}_2 &=
        \begin{bmatrix}
            \alpha_{1,1} \boldsymbol{\alpha}^1 \\
            \vdots \\
            \alpha_{1,d} \boldsymbol{\alpha}^d
        \end{bmatrix} \\
    \mathbf{T}_2 &= \begin{bmatrix}
        \mathbf{T}^1 & & \\
        & \ddots & \\
        & & \mathbf{T}^d
    \end{bmatrix} + (\mathbf{M}_K \mathbf{T}_1 \mathbf{M}_K^\top) \odot
        \Bigg( \begin{bmatrix}
            \boldsymbol{\omega}_1 \\
            \vdots \\
            \boldsymbol{\omega}_d
        \end{bmatrix} \begin{bmatrix}
            \boldsymbol{\alpha}_1\\
            \vdots \\
            \boldsymbol{\alpha}_1
        \end{bmatrix}^\top \Bigg)  \\
    \boldsymbol{\omega}_2 &= \
        \begin{bmatrix}
            \omega_{1,1} \boldsymbol{\omega}^1 \\
            \vdots \\
            \omega_{1,d} \boldsymbol{\omega}^d
        \end{bmatrix} \\
    \boldsymbol{\lambda}_2 &= \mathbf{M}_L^\top \begin{bmatrix}
        \lambda_{1,1} \circ \boldsymbol{\lambda}^1 \\
        \vdots \\
        \lambda_{1,d} \circ \boldsymbol{\lambda}^d \\
    \end{bmatrix},
\end{align}
$\mathbf{M}_K$ is a matrix whose elements belong to the semiring $K$
and it is defined as:
\begin{align}
    \mathbf{M}_K = \begin{bmatrix}
        \mathbf{1}_{|Q_1|} & & & \\
        & \mathbf{1}_{|Q_2|} & & \\
        & & \ddots & \\
        & & & \mathbf{1}_{|Q_d|}
    \end{bmatrix}.
\end{align}
where $\mathbf{1}_{|Q_i|}$ is a vector of $1$ of size $|Q_i|$. $\mathbf{M}_L$
is defined identically but has elements in the semiring $L$.

\subsection{Weight propagation}

Propagate the weights through the FSM. This operation is needed
before determinization to guarantee that the output FSM will be
equivalent to the input one.

\subsection{Determinization}

Standard powerset construction algorithm.

