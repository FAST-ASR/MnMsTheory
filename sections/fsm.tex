\section{Weighted Finite State Machine}

We define a Finite State Machine (FSM) $\mathcal{M}$ by the tuple
$\mathcal{M} = (Q, \Sigma, K, L, \boldsymbol{\alpha},~\mathbf{T},~\boldsymbol{\omega}, \boldsymbol{\lambda})$
where:
\begin{itemize}
    \item $Q = \{1, \dots, d\}$ is the set of states (with cardinality $d$)
        identified as integers
    \item $\Sigma$ is a set of symbols
    \item $K$ is an zero-sum free and ordered semiring for the FSM's weights
    \item $L$ is an union-concatenation semiring defined over $\{ S : S \subseteq \Sigma^* \}$ for the FSM's labels
    \item $\boldsymbol{\alpha} \in K^d$ is a vector such that $\alpha_i >_K 0_k$
    is the initial weight of the state~$i~\in~Q$
    \item $\mathbf{T} \in K^{d\times d}$ is a matrix such that $T_{ij}$
        is the transition weight from the state $i \in Q$ to the state~$j~\in~Q$
    \item $\boldsymbol{\omega} \in K^d$ is a vector such that $\omega_i$
        is the final weight of the state~$i~\in~Q$
    \item $\boldsymbol{\lambda} \in L^d$ is a vector of symbol such that
        $\lambda_i$ is the symbol of the state~$i~\in~Q$
\end{itemize}

\paragraph{initial and final states:} We say that a state $i$ is an
\emph{initial state} if its initial weight is greater than $0$,
i.e $\alpha_i > 0$. Similarly, we say that a state $i$ is a
\emph{final state} if its final weight is greater than $0$, i.e.
$\omega_i > 0$.

\paragraph{path:} A path
\begin{align}
    \boldsymbol{\pi} = (\pi_1, \pi_2, \dots, \pi_n), \quad
        \pi_i \in Q, \quad \forall i \in \{1, \dots, n \}
\end{align}
is a sequence of states. The weight of a path $\boldsymbol{\pi}$ is
given by the function $\mu: Q^n \rightarrow K$:
\begin{align}
    \mu(\boldsymbol{\pi}) = \alpha_{\pi_1} \Big[
        \prod_{i=2}^{n-1} T_{\pi_{i-1},\pi_i} \Big] \omega_{\pi_N}.
\end{align}
Similarly, the label sequence of the path $\boldsymbol{\pi}$ is given
by the function $\sigma: Q^n \rightarrow L^n$:
\begin{align}
    \sigma(\boldsymbol{\pi}) = \prod_{n=1}^N \lambda_{\pi_n}.
\end{align}

\paragraph{input sequence:} We define an input sequence $\mathbf{s}$ to
a FSM as a sequence of labels:
\begin{align}
    \mathbf{s} = \prod_{i=1}^n s_i, \quad s_i \in L.
\end{align}
The weight of an input sequence is given by the function
$\nu: L^n \rightarrow K$:
\begin{align}
    \nu(\mathbf{s}) = \sum_{ \mathbf{x} \in P} \mu(\mathbf{x}),
\end{align}
where $P = \{\boldsymbol{\pi} : \; \sigma(\boldsymbol{\pi}) = \mathbf{s}, \;
\boldsymbol{\pi} \in \mathcal{M} \}$ is the set of paths from
$\mathcal{M}$ with label sequence $\mathbf{s}$.
We say that an input sequence is \emph{accepted by $\mathcal{M}$} if
$\nu(\mathbf{s}) > 0$. We denote $S$ the set of accepted input sequence
of a FSM, i.e. $S = \{\mathbf{s} : \nu(\mathbf{s}) > 0 \}$.



