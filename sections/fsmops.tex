\section{FSM operations}
\label{sec:fsmops}

We describe here some operations over FSMs that have the same weights and labels
semiring $K$ and $L$. We write $\mathcal{M}_m$ the FSM defined as

\begin{equation}
    \mathcal{M}_m = (Q_m, \Sigma, K, L, \boldsymbol{\alpha}_m,
        \mathbf{T}_m, \boldsymbol{\omega}_m, \boldsymbol{\lambda}_m);
\end{equation}
$\mathcal{M}'$ is defined similarly. Finally, we write $\Pi_m$ (respectively
$\Pi'$) the set of all strings and their associated weight generated by the FSM
$\mathcal{M}$ (respectively $\mathcal{M}'$).

\subsection{Renormalization}

\subsection{Union}

The union of two FSMs $\mathcal{M}' = \mathcal{M}_1 \cup \mathcal{M}_2$ gives
a FSM such that $\Pi' = \Pi_1 \cup \Pi_2$. $\mathcal{M}'$ can be obtained with
the following construction:
\begin{align}
    Q' &= \{1, \dots, d_1, d_1 + 1, \dots, d_1 + d_2 \} \\
    \boldsymbol{\alpha}' &= \begin{bmatrix}
        \boldsymbol{\alpha}_1 \\
        \boldsymbol{\alpha}_2
    \end{bmatrix} \\
    \mathbf{T}' &= \begin{bmatrix}
        \mathbf{T}_1 & \\
        & \mathbf{T}_2
    \end{bmatrix} \\
    \boldsymbol{\omega}' &= \begin{bmatrix}
        \boldsymbol{\omega}_1 \\
        \boldsymbol{\omega}_2
    \end{bmatrix}\\
    \boldsymbol{\lambda}' &= \begin{bmatrix}
        \boldsymbol{\lambda}_1 \\
        \boldsymbol{\lambda}_2
    \end{bmatrix}
\end{align}

\subsection{Concatenation}

The concatenation of two FSMs $\mathcal{M}' = \text{concat}({M}_1, \mathcal{M}_2)$ gives a FSM such that $\Pi' = \{\pi_1 \pi_2 : \pi_1 \in \Pi_1, \ \pi_2 \in \Pi_2$. $\mathcal{M}'\}$
can be obtained with the following construction:
\begin{align}
    Q' &= \{1, \dots, d_1, d_1 + 1, \dots, d_1 + d_2 \} \\
    \boldsymbol{\alpha}' &= \begin{bmatrix}
        \boldsymbol{\alpha}_1 \\
        0_K \boldsymbol{\alpha}_2
    \end{bmatrix} \\
    \mathbf{T}' &= \begin{bmatrix}
        \mathbf{T}_1 & \\
        & \mathbf{T}_2
    \end{bmatrix} \\
    \boldsymbol{\omega}' &= \begin{bmatrix}
        0_K \boldsymbol{\omega}_1 \\
        \boldsymbol{\omega}_2
    \end{bmatrix}\\
    \boldsymbol{\lambda}' &= \begin{bmatrix}
        \boldsymbol{\lambda}_1 \\
        \boldsymbol{\lambda}_2
    \end{bmatrix}
\end{align}

\subsection{Reversal}

The reversal (denoted $^\top$) of a FSM $\mathcal{M}$ yields a FSM
$\mathcal{M}' = \mathcal{M}^\top$ such that $\Pi' = \{ \overleftarrow{\pi} : \pi \in \Pi \}$ where $\overleftarrow{\pi}$ is the reversed of $\pi$.
\begin{align}
    Q' &= Q \\
    \boldsymbol{\alpha}' &= \boldsymbol{\omega} \\
    \mathbf{T}' &= \mathbf{T}^\top \\
    \boldsymbol{\omega}' &= \boldsymbol{\alpha} \\
    \boldsymbol{\lambda}' &= \boldsymbol{\lambda}
\end{align}

